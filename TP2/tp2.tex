\documentclass[a4paper,10pt]{article}
\usepackage[paper=a4paper, hmargin=1.5cm, bottom=1.5cm, top=3.5cm]{geometry}
\usepackage[latin1]{inputenc}
\usepackage[T1]{fontenc}
\usepackage[spanish]{babel}
\usepackage{xspace}
\usepackage{xargs}
\usepackage{ifthen}
\usepackage{aed2-tad,aed2-symb,aed2-itef}


\newcommand{\moduloNombre}[1]{\textbf{#1}}

\let\NombreFuncion=\textsc
\let\TipoVariable=\texttt
\let\ModificadorArgumento=\textbf
\newcommand{\res}{$res$\xspace}
\newcommand{\tab}{\hspace*{7mm}}

\newcommandx{\TipoFuncion}[3]{%
  \NombreFuncion{#1}(#2) \ifx#3\empty\else $\to$ \res\,: \TipoVariable{#3}\fi%
}
\newcommand{\In}[2]{\ModificadorArgumento{in} \ensuremath{#1}\,: \TipoVariable{#2}\xspace}
\newcommand{\Out}[2]{\ModificadorArgumento{out} \ensuremath{#1}\,: \TipoVariable{#2}\xspace}
\newcommand{\Inout}[2]{\ModificadorArgumento{in/out} \ensuremath{#1}\,: \TipoVariable{#2}\xspace}
\newcommand{\Aplicar}[2]{\NombreFuncion{#1}(#2)}

\newlength{\IntFuncionLengthA}
\newlength{\IntFuncionLengthB}
\newlength{\IntFuncionLengthC}
%InterfazFuncion(nombre, argumentos, valor retorno, precondicion, postcondicion, complejidad, descripcion, aliasing)
\newcommandx{\InterfazFuncion}[9][4=true,6,7,8,9]{%
  \hangindent=\parindent
  \TipoFuncion{#1}{#2}{#3}\\%
  \textbf{Pre} $\equiv$ \{#4\}\\%
  \textbf{Post} $\equiv$ \{#5\}%
  \ifx#6\empty\else\\\textbf{Complejidad:} #6\fi%
  \ifx#7\empty\else\\\textbf{Descripci�n:} #7\fi%
  \ifx#8\empty\else\\\textbf{Aliasing:} #8\fi%
  \ifx#9\empty\else\\\textbf{Requiere:} #9\fi%
}

\newenvironment{Interfaz}{%
  \parskip=2ex%
  \noindent\textbf{\Large Interfaz}%
  \par%
}{}

\newenvironment{Representacion}{%
  \vspace*{2ex}%
  \noindent\textbf{\Large Representaci�n}%
  \vspace*{2ex}%
}{}

\newenvironment{Algoritmos}{%
  \vspace*{2ex}%
  \noindent\textbf{\Large Algoritmos}%
  \vspace*{2ex}%
}{}


\newcommand{\Titulo}[1]{
  \vspace*{1ex}\par\noindent\textbf{\large #1}\par
}

\newenvironmentx{Estructura}[2][2={estr}]{%
  \par\vspace*{2ex}%
  \TipoVariable{#1} \textbf{se representa con} \TipoVariable{#2}%
  \par\vspace*{1ex}%
}{%
  \par\vspace*{2ex}%
}%

\newboolean{EstructuraHayItems}
\newlength{\lenTupla}
\newenvironmentx{Tupla}[1][1={estr}]{%
    \settowidth{\lenTupla}{\hspace*{3mm}donde \TipoVariable{#1} es \TipoVariable{tupla}$($}%
    \addtolength{\lenTupla}{\parindent}%
    \hspace*{3mm}donde \TipoVariable{#1} es \TipoVariable{tupla}$($%
    \begin{minipage}[t]{\linewidth-\lenTupla}%
    \setboolean{EstructuraHayItems}{false}%
}{%
    $)$%
    \end{minipage}
}

\newcommandx{\tupItem}[3][1={\ }]{%
    %\hspace*{3mm}%
    \ifthenelse{\boolean{EstructuraHayItems}}{%
        ,#1%
    }{}%
    \emph{#2}: \TipoVariable{#3}%
    \setboolean{EstructuraHayItems}{true}%
}

\newcommandx{\RepFc}[3][1={estr},2={e}]{%
  \tadOperacion{Rep}{#1}{bool}{}%
  \tadAxioma{Rep($#2$)}{#3}%
}%

\newcommandx{\Rep}[3][1={estr},2={e}]{%
  \tadOperacion{Rep}{#1}{bool}{}%
  \tadAxioma{Rep($#2$)}{true \ssi #3}%
}%

\newcommandx{\Abs}[5][1={estr},3={e}]{%
  \tadOperacion{Abs}{#1/#3}{#2}{Rep($#3$)}%
  \settominwidth{\hangindent}{Abs($#3$) \igobs #4: #2 $\mid$ }%
  \addtolength{\hangindent}{\parindent}%
  Abs($#3$) \igobs #4: #2 $\mid$ #5%
}%

\newcommandx{\AbsFc}[4][1={estr},3={e}]{%
  \tadOperacion{Abs}{#1/#3}{#2}{Rep($#3$)}%
  \tadAxioma{Abs($#3$)}{#4}%
}%


\newcommand{\DRef}{\ensuremath{\rightarrow}}

\begin{document}

\section{M�dulo Ciudad Robotica}


\begin{Interfaz}


  \textbf{se explica con}: \tadNombre{Secuencia$(\alpha)$}, \tadNombre{Iterador Bidireccional($\alpha$)}.

  \textbf{g�neros}: \TipoVariable{ciudad}, \TipoVariable{itLista($\alpha$)}.

  \textbf{usa}: 
  
  \Titulo{Operaciones b�sicas de ciudad}

  \InterfazFuncion{Crear}{\In{m}{mapa}}{ciudad}%
  {$res \igobs$ crear($m$)}%
  [$\Theta(1)$]
  [genera una nueva Ciudad.]
  [lo ideal seria no copiar ese mapa no?]
  
  \InterfazFuncion{Entrar}{\In{ts}{conj(tag)}, \In{e}{estacion}, \Inout{c}{ciudad}}{}
  [$c \igobs c_0 \land e \in$ estaciones($c$)]
  {$c \igobs$ entrar($ts, e, c_0$)}
  [$\Theta(copy(a))$]
  [...]
  [...]

  \InterfazFuncion{Mover}{\In{u}{rur}, \In{e}{estacion}, \Inout{c}{ciudad}}{}
  [$c \igobs c_0 \land e \in$ estaciones($c$)$\land \in $ robots($c$)]
  {$c \igobs$ mover($u, e, c_0$)}
  [$\Theta(copy(a))$]
  [...]
  [...]
  
  \InterfazFuncion{Inspeccion}{\In{e}{estacion}, \Inout{c}{ciudad}}{}
  [$c \igobs c_0 \land e \in$ estaciones($c$)]
  {$c \igobs$ inspeccion($e, c_0$)}
  [$\Theta(copy(a))$]
  [...]
  [...]
 
 \InterfazFuncion{ProximoRUR}{\In{c}{ciudad}}{rur}
  {$res \igobs$ proximoRUR($c$)}
  [$O(1)$]
  [...]
  [...]
  
  \InterfazFuncion{Mapa}{\In{c}{ciudad}}{mapa}
  {$res \igobs$ mapa($c$)}
  [$\Theta(copy(a))$]
  [...]
  [...]
  
  \InterfazFuncion{Robots}{\In{c}{ciudad}}{it(conj(rur))}
  {$res \igobs$ CrearIt(robots($c$))}
  [$O(1)$]
  [...]
  [...]
  
  \InterfazFuncion{Estacion}{\In{u}{rur}, \In{c}{ciudad}}{estacion}
  [$u \in$  robots($c$)]
  {$res \igobs$ estacion($u, c$)}
  [$O(1)$]
  [...]
  [...]
  
  \InterfazFuncion{Tags}{\In{u}{rur}, \In{c}{ciudad}}{conj(tag)}
  [$u \in$ robots($c$)]
  {$c \igobs$ inspeccion($e, c_0$)}
  [$O(1)$]
  [...]
  [...]

  \InterfazFuncion{$\#$Infracciones}{\In{u}{rur}, \In{c}{ciudad}}{nat}
  [$u \in$ robots($c$)]
  {$c \igobs$ inspeccion($e, c_0$)}
  [$\Theta(copy(a))$]
  [...]
  [...]

  \InterfazFuncion{Estaciones}{\In{c}{ciudad}}{it(conj(estaciones))}
  {$res \igobs$ CrearIt(estaciones($c$))} %CrearIt es la del modulo conjunto
  [$O(1)$]
  [...]
  [...]
  
\end{Interfaz}

\begin{Representacion}
  
  \Titulo{Representaci�n de la ciudad}

  \begin{Estructura}{ciudad}[str]
    \begin{Tupla}[str]
      \tupItem{robRUR}{DiccArreglo(rur, puntero(datosRobot))}%
      \tupItem{\\robEstacion}{DiccTrie(estacion, colaPrio(rur))}%
      \tupItem{\\mapa}{vector(senda)}%
    \end{Tupla}

    \begin{Tupla}[senda]
	 \tupItem{est1}{estacion}
	 \tupItem{est2}{estacion}
	 \tupItem{restr}{restriccion}   
    \end{Tupla}
    
    \begin{Tupla}[datosRobot]
      \tupItem{presente?}{bool}%
      \tupItem{\\est}{estacion}%
      \tupItem{\\infr}{nat}%
      \tupItem{\\tags}{conjTrie(tags)}%
      \tupItem{\\permisos}{DiccTrie(estacion, DiccTrie(estacion, bool)}%
      \tupItem{\\itEst}{it(colaPrio(rur))}%
    \end{Tupla}
   
    
  \end{Estructura}

  \Rep[lst][l]{($l$.primero $=$ NULL) $=$ ($l$.longitud $=$ $0$) $\yluego$ ($l$.longitud $\neq$ $0$ \impluego \\
    Nodo($l$, $l$.longitud) $=$ $l$.primero $\land$ \\
    ($\forall i$: nat)(Nodo($l$,$i$)\DRef siguiente $=$ Nodo($l$,$i+1$)\DRef anterior) $\land$ \\
    ($\forall i$: nat)($1 \leq i <$ $l$.longitud $\implies$ Nodo($l$,$i$) $\neq$ $l$.primero)}\mbox{}

  ~      

  \tadOperacion{Nodo}{lst/l,nat}{puntero(nodo)}{$l$.primero $\neq$ NULL}
  \tadAxioma{Nodo($l$,$i$)}{\IF $i = 0$ THEN $l$.primero ELSE Nodo(FinLst($l$), $i-1$) FI}

  ~

  \tadOperacion{FinLst}{lst}{lst}{}
  \tadAxioma{FinLst($l$)}{Lst($l$.primero\DRef siguiente, $l$.longitud $-$ $\min$\{$l$.longitud, $1$\})}

  ~

  \tadOperacion{Lst}{puntero(nodo),nat}{lst}{}
  \tadAxioma{Lst($p,n$)}{$\langle p, n\rangle$}

  ~
 
  \AbsFc[lst]{secu($\alpha$)}[l]{\IF $l$.longitud $=$ $0$ THEN \secuencia{} ELSE \secuencia{$l$.primero\DRef dato}[Abs(FinLst($l$))] FI}

  \Titulo{Representaci�n del iterador}

  \begin{Estructura}{itLista($\alpha$)}[iter]
    \begin{Tupla}[iter]
      \tupItem{siguiente}{puntero(nodo)}%
      \tupItem{lista}{puntero(lst)}%
    \end{Tupla}
  \end{Estructura}

  \Rep[iter][it]{Rep($\ast$($it$.lista)) $\yluego$ ($it$.siguiente $=$ NULL $\oluego$ ($\exists i$: nat)(Nodo($\ast it$.lista, $i$) $=$ $it$.siguiente)}

  ~

  \Abs[iter]{itBi($\alpha$)}[it]{b}{Siguientes($b$) $=$ Abs(Sig($it$.lista, $it$.siguiente)) $\land$\\
    Anteriores($b$) $=$ Abs(Ant($it$.lista, $it$.siguiente))}

  ~

  \tadOperacion{Sig}{puntero(lst)/l,puntero(nodo)/p}{lst}{Rep($\langle l, p\rangle$)}
  \tadAxioma{Sig($i, p$)}{Lst($p$, $l$\DRef longitud $-$ Pos($\ast l$, $p$))}

  ~

  \tadOperacion{Ant}{puntero(lst)/l,puntero(nodo)/p}{lst}{Rep($\langle l, p\rangle$)}
  \tadAxioma{Ant($i, p$)}{Lst(\IF $p$ $=$ $l$\DRef primero THEN NULL ELSE $l$\DRef primero FI, Pos($\ast l$, $p$))}

  ~

  {\small Nota: cuando $p$ $=$ NULL, Pos devuelve la longitud de la lista, lo cual est� bien, porque significa que el iterador no tiene siguiente.}
  \tadOperacion{Pos}{lst/l,puntero(nodo)/p}{puntero(nodo)}{Rep($\langle l, p\rangle$)}
  \tadAxioma{Pos($l$,$p$)}{\IF $l$.primero $=$ $p$ $\lor$ $l$.longitud $=$ $0$ THEN $0$ ELSE $1$ $+$ Pos(FinLst($l$), $p$) FI}


\end{Representacion}

\begin{Algoritmos}
 
 
\end{Algoritmos}

\section{Modulo Cola de Prioridad}

\begin{Interfaz}
  \textbf{se explica con}: \tadNombre{Cola de prioridad($\alpha$)}

  \textbf{g�neros}: \TipoVariable{colaPrio($\alpha$)}

  \textbf{usa}: 

  \InterfazFuncion{Vacia}{}{colaPrio($\alpha$)}%
  {$res \igobs$ vacia}%
  [$\Theta(1)$]
  [genera una cola vacia.]
  
  \InterfazFuncion{Encolar}{\In{a}{$\alpha$}, \In{n}{nat},\Inout{c}{colaPrio($\alpha$)}}{}%
  [$c \igobs c_0$]
  {$res \igobs$ encolar(a, c)}%
  [$\Theta(1)$]
  [agrega a la cola el elemento $a$]
  [$a$ se agrega por copia]
  
    
  \InterfazFuncion{Vacia?}{\In{c}{colaPrio($\alpha$)}}{bool}%
  {$res \igobs$ vacia?(c)}%
  [$\Theta(1)$]
  [checkea si la cola esta vacia]
  
  \InterfazFuncion{Desencolar}{\Inout{c}{colaPrio($\alpha$)}}{$\alpha$}%
  [$c \igobs c_0 \land \neg$ vacia?($c_0$)]%
  {}
  [$O(log n)$]
  [elimina el proximo de la cola y retorna el elemento]
  []  
  
    \Titulo{Operaciones del iterador}
El iterador es necesario para eliminar en el mover
  
\end{Interfaz}

\begin{Representacion}
\Titulo{Representacion de la cola}
\begin{Estructura}{colaPrio($\alpha$)}[c]
  \begin{Tupla}[c]
   \tupItem{raiz}{puntero(Nodo)}
   \tupItem{padreUlt}{puntero(Nodo)}
   \tupItem{cant}{nat}
  \end{Tupla}
  
  \begin{Tupla}[Nodo]
   \tupItem{prioridad}{nat}
   \tupItem{\\elem}{$\alpha$}   
   \tupItem{\\padre}{puntero(Nodo)}
   \tupItem{\\izq}{puntero(Nodo)}
   \tupItem{\\der}{puntero(Nodo)}
  \end{Tupla}


\end{Estructura}

\end{Representacion}
\begin{Algoritmos}\\
\TipoFuncion{iVacia}{}{colaPrio($\alpha$)}\\
\indent res.raiz $\leftarrow$ NULL\\
\indent res.cant $\leftarrow$ 0\\
\indent res.padreUlt $\leftarrow$ NULL\\
\\
\TipoFuncion{iEncolar}{\Inout{c}{colaPrio($\alpha$)}, \In{a}{$\alpha$}, \In{n}{nat}}{}\\
\indent c.cant++ \\
\indent var n $\leftarrow$ New Nodo\\
\indent n.prioridad $\leftarrow$ n\\
\indent n.elem $\leftarrow$ a\\
\indent if c.raiz = NULL \{ c.raiz $\leftarrow$ \& n\} \\
\indent else \{\\ 
\indent \indent if c.padreUlt $==$ NULL\{\\
\indent \indent c.padreUlt $\leftarrow$ c.raiz\\ 
\indent \indent n.padre $\leftarrow$ c.raiz\\
\indent \indent c.raiz$\rightarrow$izq $\leftarrow$ \& n \\
\indent \indent \} else \{\\
\indent \indent \indent if c.padreUlt$\rightarrow$der $==$ NULL\{\\
\indent \indent \indent \indent n.padre $\leftarrow$ c.padreUlt\\
\indent \indent \indent \indent c.padreUlt$\rightarrow$der $\leftarrow$ \& n\\
\indent \indent \indent \}else\{ \\
\indent \indent \indent \indent if $log_2(cant) == int(log_2(cant))$ \{\\
\indent \indent \indent \indent \indent c.padreUlt $\leftarrow$ busqIzq(c)\\
\indent \indent \indent \indent \indent n.padre $\leftarrow$ c.padreUlt\\
\indent \indent \indent \indent \indent c.padreUlt$\rightarrow$izq $\leftarrow$ \& n\\
\indent \indent \indent \indent \}else\{\\
\indent \indent \indent \indent \indent c.padreUlt $\leftarrow$ busqTransversal(c.padreUlt, c)\\
\indent \indent \indent \indent \indent n.padre $\leftarrow$ c.padreUlt\\
\indent \indent \indent \indent \indent c.padreUlt$\rightarrow$izq $\leftarrow$ \& n\\
\indent \indent \indent \indent \}\\
\indent \indent \indent \}\\
\indent \indent \}\\
\indent \}\\
\indent var parent$\leftarrow$ n.padre\\
\indent var aux\\
\indent while(parent != NULL $\land$ n.prioridad > parent$\rightarrow$prioridad)\{\\
\indent \indent if parent$\rightarrow$der == n\{\\
\indent \indent \indent n.padre $\leftarrow$ parent$\rightarrow$padre\\
\indent \indent \indent aux $\leftarrow$ parent$\rightarrow$izq\\
\indent \indent \indent parent$\rightarrow$der $\leftarrow$ n.der\\
\indent \indent \indent parent$\rightarrow$izq $\leftarrow$ n.izq\\
\indent \indent \indent n.der $\leftarrow$ parent\\
\indent \indent \indent n.izq $\leftarrow$ aux\\
\indent \indent \indent parent$\rightarrow$padre $\leftarrow$ \& n\\
\indent \indent \} else \{\\
\indent \indent \indent n.padre $\leftarrow$ parent$\rightarrow$padre\\
\indent \indent \indent aux $\leftarrow$ parent$\rightarrow$der\\
\indent \indent \indent parent$\rightarrow$der $\leftarrow$ n.der\\
\indent \indent \indent parent$\rightarrow$izq $\leftarrow$ n.izq\\
\indent \indent \indent n.izq $\leftarrow$ parent\\
\indent \indent \indent n.der $\leftarrow$ aux\\
\indent \indent \indent parent$\rightarrow$padre $\leftarrow$ \& n\\
\indent \indent \}\\
\indent \}\\
\indent res $\leftarrow$ crearIt(\& n)\\
\\
\TipoFuncion{ibusqIzq}{\In{c}{colaPrio($\alpha$)}}{Nodo}\\
\indent var actual $\leftarrow$ c.raiz\\
\indent while(actual$\rightarrow$izq != NULL)\{\\
\indent \indent actual $\leftarrow$ actual$\rightarrow$izq\\
\indent \}\\
\indent res $\leftarrow$ actual\\
\\
\TipoFuncion{ibusqTransversal}{\In{c}{colaPrio($\alpha$)}, \In{n}{Nodo}}{Nodo}\\
\indent var parent $\leftarrow$ n.padre\\
\indent var act $\leftarrow$ n\\
\indent while(parent$\rightarrow$der == act)\{\\
\indent \indent act $\leftarrow$ parent\\
\indent \indent parent $\leftarrow$ parent$\rightarrow$padre\\
\indent \}\\
\indent act $\leftarrow$ parent$\rightarrow$der\\
\indent while(act$\rightarrow$izq =! NULL)\{\\
\indent \indent act $\leftarrow$ act$\rightarrow$izq\\
\indent \}\\
\indent res $\leftarrow$ act\\
\\
\TipoFuncion{iVacia?}{\In{c}{colaPrio($\alpha$)}}{bool}\\
\indent res $\leftarrow$ c.cant $==$ 0\\
\\
\TipoFuncion{iDesencolar}{\Inout{c}{colaPrio($\alpha$)}}{}\\
\indent var ult\\
\indent if c.padreUlt$\rightarrow$der != NULL \{\\
\indent \indent ult $\leftarrow$ c.padreUlt$\rightarrow$der\\
\indent \indent c.padreUlt$\rightarrow$der $\leftarrow$ NULL\\
\indent \}else \{\\
\indent \indent ult $\leftarrow$ c.padreUlt$\rightarrow$izq\\
\indent \indent c.padreUlt$\rightarrow$izq $\leftarrow$ NULL\\
\indent \}\\
\indent ult$\rightarrow$padre $\leftarrow$ NULL\\
\indent ult$\rightarrow$izq $\leftarrow$ c.raiz$\rightarrow$izq\\
\indent ult$\rightarrow$der $\leftarrow$ c.raiz$\rightarrow$der\\
\indent ult$\rightarrow$der$\rightarrow$padre $\leftarrow$ ult\\
\indent ult$\rightarrow$izq$\rightarrow$padre $\leftarrow$ ult\\
\indent delete c.raiz\\
\indent c.raiz $\leftarrow$ ult\\
\indent bajar

\end{Algoritmos}

\section{Modulo DiccArreglo(nat, significado)}

\begin{Interfaz}
  \textbf{par�metros formales}\hangindent=2\parindent\\
  \parbox{1.7cm}{\textbf{g�neros}} nat, significado\\
  
  \textbf{se explica con}: \tadNombre{dicc($clave, significado$)}, \tadNombre{Iterador Unidireccional($\alpha$)}.\\
  \indent\textbf{g�neros}: \TipoVariable{DiccArreglo(nat, significado)}, \TipoVariable{itDicc(nat)}.

  \Titulo{Operaciones b�sicas del diccionario arreglo}
  
  \InterfazFuncion{Vacio}{}{DiccArreglo(nat, significado)}%
  {$res \igobs$ vacio()}%
  [$\Theta(1)$]
  [genera un nuevo diccionario arreglo.]
  
  \InterfazFuncion{Definir}{\Inout{DA}{DiccArreglo(nat, significado)}, \In{s}{significado}}{}%
  [$DA_0 \igobs DA$]
  {$DA \igobs$ definir($\#claves(DA_0) + e$, s, $DA_0$)}%
  [$\Theta(n)$]
  [Define una nueva clave con su significado. Donde e es la cantidad de elementos eliminados del diccionario 	historicamente\\
   n es la cantidad de elementos en el diccionario en el estado anterior a Definir.] 

  \InterfazFuncion{Def?}{\In{a}{nat}, \In{DA}{DiccArreglo(nat, significado)}}{bool}%
  {$res \igobs$ def?(a, DA)}%
  [$\Theta(1)$]
  [Devuelve $True$ si el la clave est� definida.]

  \InterfazFuncion{Obtener}{\In{a}{nat}, \In{DA}{DiccArreglo(nat, significado)}}{significado}%
  [Def?(a, DA)]  
  {$res \igobs$ obtener(a, DA)}%
  [$\Theta(1)$]
  [Devuelve el significado de la clave a.]
  
  \InterfazFuncion{Borrar}{\In{a}{nat}, \Inout{DA}{DiccArreglo(nat, significado)}}{}%
  [Def?(a, DA) $\land$ $DA_0 \igobs DA$]  
  {$DA \igobs$ borrar(a, $DA_0$)}%
  [$\Theta(1)$]
  [Borra el significado y su clave a.]
  
  \InterfazFuncion{Claves}{\Inout{DA}{DiccArreglo(nat, significado)}}{itDicc(nat)}%  
  {$res \igobs$ CrearItUni(convertir(claves($DA$)))}%
  [$\Theta(1)$]
  [Devuelve el conjunto de claves del diccionario.]
  
  \Titulo{Operaciones b�sicas del iterador}
  
  \InterfazFuncion{CrearIt}{\In{DA}{DiccArreglo(nat, significado)}}{itDicc(nat)}%  
  {$res$ $\leftarrow$ CrearItUni(convertir(claves($DA$)))}%
  [$\Theta(1)$]
  [Crea un iterador unidireccional del conjunto de claves \\
  de forma tal que siguiente devuelva la siguiente clave del diccionario.]
    
  \InterfazFuncion{Avanzar}{\Inout{it}{itDicc(nat)}}{}%  
  [$it = it_0$ $\land$ haySiguiente?(it)]
  {$it \igobs$ avanzar($it_0$)}%
  [$\Theta(1)$]
  [Avanza a la posici�n siguiente del iterador.]  
  
  \InterfazFuncion{Siguiente}{\In{it}{itDicc(nat)}}{nat}%  
  [haySiguiente?(it)]
  {$res \igobs$ siguiente($it$)}%
  [$\Theta(1)$]
  [devuelve el elemento siguiente a la posici�n del iterador.] 
  [res no es modificable.]  
  
  \InterfazFuncion{HaySiguiente?}{\In{it}{itDicc(nat)}}{bool}%  
  {$res \igobs$ haySiguiente($it$)}%
  [$\Theta(1)$]
  [devuelve $true$ si y solo si en el iterador quedan elementos para avanzar.] 
  [res no es modificable.]  
    
\end{Interfaz}

\begin{Representacion}

\Titulo{Representaci�n del diccionario arreglo}
El objetivo de este modulo es implementar un arreglo de elementos que son una tupla con un significado y un booleano.
La idea es que si el booleano vale True significa que el elemento est� borrado. 

 \begin{Estructura}{DiccArreglo(nat, significado)}[vec : vector(valor)]
  \begin{Tupla}[valor]
   \tupItem{sig}{significado}
   \tupItem{esta?}{bool}
  \end{Tupla}
 \end{Estructura}

  \Rep[vec][v]{true}\mbox{}\

  \AbsFc[vector(valor)]{dicc(nat, significado)}[vec]{\IF Longitud($vec$) $=$ $0$ THEN vacio ELSE 
  \textbf{if} $vec$[Longitud($vec$) $-$ 1].esta? = $true$ \textbf{then}\\ definir(Longitud($vec$) $-$ 1, $vec$[Longitud($vec$) $-$ 1].significado, AbsFc(Comienzo($vec$))) \\\textbf{else} \\
   AbsFc(Comienzo($vec$)) \\\textbf{fi} FI} 


\Titulo{Representaci�n del iterador}
Este iterador recorre las claves que son los naturales en el rango del vector. Para esto mantiene una variable\\
actualizada con la posici�n siguiente. Este iterador se indefine cada vez que el vector se redefine. En ese caso debera crearse un nuevo iterador.

 \begin{Estructura}{itDicc(nat)}[itTupla]
  \begin{Tupla}[itTupla]
   \tupItem{posicion}{nat}
   \tupItem{pVec}{puntero(vector(valor))}   
  \end{Tupla}
 \end{Estructura}


 \Rep[itTupla][it]{$\neg$($it$.pVec  $=$  NULL) $\yluego$ \\ 
 ($it$.posicion $<$ Longitud(*($i$.pVec)) \&\& 0 $\leq$ $it$.posicion) $\yluego$ \\ 
 (*($it$.pVec)[$it$.posicion].esta? = true) 
 }\mbox{} \\

  \Abs[itTupla]{itUni(nat)}[it]{iu}{Siguientes($iu$) $=$ convertir(claves(Abs(*($it$.pVec)))) $\land$\\ 	
  Siguiente($iu$) $=$ $it$.posici�n}

\end{Representacion}

\begin{Algoritmos}

  \TipoFuncion{iVacio}{}{vector(valor)} \\
  \indent\indent res $\leftarrow$ Vac�a();\\
  
  \TipoFuncion{iDefinir}{\Inout{vec}{vector(valor)}, \In{s}{significado}}{} \\
  \indent\indent\indent\indent res  $\leftarrow$ AgregarAtras(vec, s);\\
  
  \TipoFuncion{iDef?}{\In{a}{nat}, \Inout{vec}{vector(valor)}}{bool} \\
  \indent\indent res $\leftarrow$ false; \\ 
  \indent\indent if a $<$ Longitud(vec) \&\& a $>$ 0 \{ \\
  \indent\indent\indent\indent if vec[a].esta? true \{ \\ 
  \indent\indent\indent\indent\indent\indent res $\leftarrow$ true; \\
  \indent\indent\indent\indent \} \\ 
  \indent\indent\} \\

  \TipoFuncion{iObtener}{\In{a}{nat}, \Inout{vec}{vector(valor)}}{valor} \\
  \indent\indent res $\leftarrow$ vec[a].significado; \\

  \TipoFuncion{iBorrar}{\In{a}{nat}, \Inout{vec}{vector(valor)}}{} \\
  \indent\indent vec[a].esta? $\leftarrow$ false; \\ 
  
  \TipoFuncion{iClaves}{\Inout{vec}{vector(valor)}}{itDicc(nat)} \\
  \indent\indent res $\leftarrow$ iCrearIt(vec);
  
\end{Algoritmos}
\section{Modulo Restriccion}

\begin{Interfaz}

  \textbf{se explica con}: \tadNombre{Restriccion}

  \textbf{g�neros}: \TipoVariable{restriccion}

  \textbf{usa}: 


  \InterfazFuncion{NuevoTag}{\In{t}{tag}}{restriccion}%
  {$res \igobs$ <t>}%
  [$\Theta(1)$]
  [genera una restriccion de un solo tag.]

  \InterfazFuncion{Not}{\Inout{r}{restriccion}}{}% este capaz que es inout
  {$res \igobs$ NOT $r$}%
  [$\Theta(1)$]
  [...]
  [...]

  \InterfazFuncion{And}{\In{r1}{restriccion}, \In{r2}{restriccion}}{restriccion}%
  {$res \igobs$ $r1$ AND $r2$}%
  [$\Theta(1)$]
  [...]
  [...]

  \InterfazFuncion{Or}{\In{r1}{restriccion}, \In{r2}{restriccion}}{restriccion}%
  {$res \igobs$ $r1$ OR $r2$}%
  [$\Theta(1)$]
  [...]
  [...]
  
  \InterfazFuncion{Verifica?}{\In{ts}{conj(tag)}, \In{r}{restriccion}}{bool}%
  {$res \igobs$ verifica?(ts, r)}%
  [$O(R)$]
  [...]
  [...]  
  
  
\end{Interfaz}

\begin{Representacion}
 \Titulo{Representacion de la restriccion}
 
 \begin{Estructura}{restriccion}[rtr]
  \begin{Tupla}[rtr]
   \tupItem{raiz}{puntero(Nodo)}
  \end{Tupla}

  \begin{Tupla}[Nodo]
   \tupItem{tag}{tag}
   \tupItem{\\tipo}{log}
   \tupItem{\\negado?}{bool}
   \tupItem{\\izq}{puntero(Nodo)}
   \tupItem{\\der}{puntero(Nodo)}
  \end{Tupla}
  
  \indent donde log es enum\{AND, OR, CAR\}
 \end{Estructura}

\Rep[rtr][r]{completo?(r.raiz) \yluego noCiclos(r.raiz, $\emptyset$) \yluego esRestriccion?(r.raiz)}

\tadOperacion{completo?}{Nodo/n}{bool}{}
\tadAxioma{completo?(n)}{(n.izq \igobs NULL $\land$ n.der \igobs NULL) $\lor$\\
$\neg$(n.izq \igobs NULL) $\land$ $\neg$ (n.der \igobs NULL) $\land$\\
completo?(n.izq) $\land$ completo?(n.der)}

\tadOperacion{esRestriccion?}{Nodo/n}{bool}{}
\tadAxioma{esRestriccion?(n)}{if n.tipo \igobs CAR \\ 
then n.izq \igobs NULL $\land$ n.der \igobs NULL\\
else n.izq $\neq$ NULL $\land$ n.der $\neq$ NULL \yluego \\
esRestriccion?(n.izq) $\land$ esRestriccion?(n.der)\\ end if}

\Abs[rtr]{restriccion}[r]{rest}{($\forall$ ts : conj(tag)) verifica?(ts, rest) $=$ Verifica?(ts, r)}

\end{Representacion}

\begin{Algoritmos}\\
\TipoFuncion{iNuevoTag}{\In{t}{tag}}{restriccion}\\ 
 \indent puntero(Nodo) n $\leftarrow$ New Nodo\\
 \indent n.tag $\leftarrow$ t\\
 \indent n.tipo $\leftarrow$ CAR\\
 \indent n.negado $\leftarrow$ false\\
 \indent n.izq $\leftarrow$ NULL\\
 \indent n.der $\leftarrow$ NULL\\
 \indent res.raiz $\leftarrow$ n\\
\\
\TipoFuncion{iNot}{\Inout{r1}{restriccion}}{}\\  
\indent r1.negado? $\leftarrow$ true\\
\\
\TipoFuncion{iAnd}{\In{r1}{restriccion}, \In{r2}{restriccion}}{restriccion}\\
 \indent puntero(Nodo) n $\leftarrow$ New Nodo\\
 \indent n.tipo $\leftarrow$ AND\\
 \indent n.negado $\leftarrow$ false\\
 \indent n.izq $\leftarrow$ r1.raiz\\
 \indent n.der $\leftarrow$ r2.raiz\\
 \indent res.raiz $\leftarrow$ n\\
\\ 
\TipoFuncion{iOr}{\In{r1}{restriccion}, \In{r2}{restriccion}}{restriccion}\\
 \indent puntero(Nodo) n $\leftarrow$ New Nodo\\
 \indent n.tipo $\leftarrow$ OR\\
 \indent n.negado $\leftarrow$ false\\
 \indent n.izq $\leftarrow$ r1.raiz\\
 \indent n.der $\leftarrow$ r2.raiz\\
 \indent res.raiz $\leftarrow$ n\\
\\
\TipoFuncion{iVerifica? }{\In{r}{restriccion}, \In{ts}{tags}}{bool}\\
\indent res $\leftarrow$ verificaAux(r.raiz, ts)
\\
\\
\TipoFuncion{iverificaAux}{\In{n}{Nodo}, \In{ts}{tags}}{bool}\\
\indent bool aux $\leftarrow$ false\\
\indent if n.tipo = CAR \\
\indent \indent then aux $\leftarrow$ pertenece?(n.tag, ts)\\
\indent \indent else if n.tipo = AND\\
\indent \indent \indent then aux $\leftarrow$ verificaAux(n.izq, ts) $\land$ verificaAux(n.der, ts)\\ 
\indent \indent \indent else aux $\leftarrow$ verificaAux(n.izq, ts) $\lor$ verificaAux(n.der, ts)\\
\indent if n.negado? then res $\leftarrow$ $\neg$ aux\\
\indent \indent else res $\leftarrow$ aux\\

\end{Algoritmos}

\newpage

\section{M�dulo Diccionario($\sigma$)}

\begin{Interfaz}
  
  \textbf{par�metros formales}\hangindent=2\parindent\\
  \parbox{1.7cm}{\textbf{g�neros}} $\sigma$\\
  % notar que ya existe el tad diccionario, este es otro. por ahora lo dejo asi, pero lo mejor seria darle un renombre.
  \textbf{se explica con}: \tadNombre{Diccionario($\sigma$)}.

  \textbf{g�neros}: \TipoVariable{dicc($\sigma$)}.

  \Titulo{Operaciones b�sicas de diccionario}

  \InterfazFuncion{Vac�o}{}{dicc($\sigma$)}
  {$res \igobs vacio$}
  [$\Theta(1)$]
  [genera un diccionario vac�o.]

  \InterfazFuncion{Definir}{\Inout{d}{dicc($\sigma$)}, \In{k}{string}, \In{s}{$\sigma$}}{}
  [$d \igobs d_0$]
  {$d = definir(d, k, s)$}
  [$\Theta(long(k))$]
  [define la clave $k$ con el significado s en el diccionario.]
  [la clave se define por copia, pero el significado se define por referencia.]

  \InterfazFuncion{Definido?}{\In{d}{dicc($\sigma$)}, \In{k}{string}}{bool}
  [long(k) $>$ 0]
  {$res \igobs def?(d,k)$}
  [$\Theta(long(k))$]
  [devuelve true si k esta definido en el diccionario.]

  \InterfazFuncion{Significado}{\In{d}{dicc($\sigma$)}, \In{k}{string}}{$\sigma$}
  [$def?(d,k)$]
  {$res \igobs obtener(k,d)$}
  [$\Theta(long(k))$]
  [devuelve la referencia al significado de la clave k en d.]

\end{Interfaz}

\begin{Representacion}
  
  \Titulo{Representaci�n del diccionario}

  \begin{Estructura}{dicc$(\sigma)$}[puntero(nodo)]
    \begin{Tupla}[nodo]
      \tupItem{caracteres}{ad(puntero(nodo))}%
      \tupItem{significado}{puntero($\sigma$)}%
    \end{Tupla}

  \end{Estructura}

  \RepFc[estr][d]{tam(d\DRef caracteres) = 256 $\yluego$ \\ ($\forall i$: nat) (n $\in$ [1..256] $\land$ d\DRef caracteres[n-1] $\neq$ NULL) $\impluego$ Rep(d\DRef caracteres[n-1])}\mbox{}

  \Abs[estr]{dicc($\sigma$)}[d]{dic}{
  ($\forall s$: string) def?(dic, s) = estaDefinido?(s, d) $\land$ \\ (def?(dic, s) \impluego ($\forall k$: string) obtener(k, dic) = dameSignificado(s, d))
  }

  ~

  \tadOperacion{dameSignificado}{string/s, puntero(nodo)/d}{bool}{}
  \tadAxioma{dameSignificado(s, d)}{
  
  \LIF{ d = NULL } \LTHEN{ NULL} \LELSE{ \\
    \LIF{ long(s) = 0} \LTHEN{ d\DRef significado }\\
    \LELSE{ dameSignificado(fin(s), d\DRef caracteres[ord(prim(s))-1])} \LFI
     } \\ \LFI
  }

  \tadOperacion{estaDefinido?}{string/s, puntero(nodo)/p}{bool}{}
  \tadAxioma{estaDefinido?(s, p)}{dameSignificado(s, p) $\neq$ NULL}

\end{Representacion}

\newpage

\begin{Algoritmos}

\TipoFuncion{iVacio}{}{dicc($\sigma$)} \\
\indent res $\leftarrow$ <caracteres: arreglo[256] de puntero(nodo), significado: puntero($\sigma$)>

~

\TipoFuncion{nuevoNodo}{}{nodo} \\
\indent res $\leftarrow$ <ad(puntero(nodo)), puntero($\sigma$)> 
%\indent $tupla.campo_1$ = arreglo[256] de puntero(nodo) \\
%\indent $tupla.campo_2$ = new $\sigma$ 

~

\TipoFuncion{iDefinir}{\Inout{d}{dicc($\sigma$)}, \In{k}{string}, \In{s}{$\sigma$}}{} \\
\indent var nodoActual $\leftarrow$ d \\
\indent var i $\leftarrow$ 0 \\
\indent while i $<$ long(k) \{ \\
\indent \indent if *nodoActual.campo$_1$[ord(k[i])] == NULL \\
\indent \indent \indent *nodoActual.campo$_1$[ord(k[i])] $\leftarrow$ nuevoNodo() \\
\indent \indent \} \\
\indent \indent nodoActual $\leftarrow$ *nodoActual.campo$_1$[ord(k[i])] \\
\indent \indent i++ \\
\indent \} \\
\indent *nodoActual.campo$_2$ $\leftarrow$ \&s

~

\TipoFuncion{iDefinido?}{\In{d}{dicc($\sigma$)}, \In{k}{string}}{bool} \\
\indent var nodoActual $\leftarrow$ d \\
\indent var i $\leftarrow$ 0 \\
\indent var seguirBuscando $\leftarrow$ true \\
\indent while i $<$ long(k) \&\& seguirBuscando \{ \\
\indent \indent if *nodoActual.campo$_1$[ord(k[i])] == NULL \{ \\
\indent \indent \indent seguirBuscando $\leftarrow$ false \\
\indent \indent \} else \{ \\
\indent \indent \indent nodoActual $\leftarrow$ *nodoActual.campo$_1$[ord(k[i])] \\
\indent \indent \indent i++ \\
\indent \indent \} \\
\indent \} \\
\indent res $\leftarrow$ false \\
\indent if i == long(k)  \{ \\
\indent \indent  res $\leftarrow$ (*nodoActual.campo$_2$[ord(k[i])] $\neq$ NULL) \\
\indent \} \\

~

\TipoFuncion{iSignificado}{\In{d}{dicc($\sigma$)}, \In{k}{string}}{$\sigma$} \\
\indent var nodoActual $\leftarrow$ d \\
\indent var i $\leftarrow$ 0 \\
\indent while i $<$ long(k) \{ \\
\indent \indent nodoActual $\leftarrow$ *nodoActual.campo$_1$[ord(k[i])] \\
\indent \indent i++ \\
\indent \} \\
\indent res $\leftarrow$ *nodoActual.campo$_2$

\end{Algoritmos}

\end{document}
